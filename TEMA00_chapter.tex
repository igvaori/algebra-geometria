\section{Intoducción.}

La confección de este texto es fruto de una larga experiencia como profesor de matemáticas de secundaria y para ello me he basado en mis más de treinta años de docencia  y en la de tantos autores que han contribuido a la explicación de estos conceptos a multitud de alumnos. He usado también apuntes y problemas de libros de texto de segundo de bachillerato: Anaya, Marea Verde, SM, Santillana, Editex, Edunsa, Alhambra, Marfil, Ecir, Bruno, etc. Así como apuntes y ejercicios encontrados en la web y pruebas de acceso a la universidad de distintas comunidades autónomas. Gracias a todos ellos por su inestimable ayuda para la confección de este pequeño texto que espero que sirva a alguien y que escribo libre de todo tipo de derechos. En particular, me han sido de mucha utilidad los `Apuntes de Álgebra' de José Salvador Cánovas Peña, profesor del departamento de Matemática Aplicada y Estadística de la Universidad Politécnica de Cartagena.

Los apartados, teoremas y los ejercicios de mayor nivel estarán marcados con el símbolo $\divideontimes$. Exceden los contenidos de un curso de segundo de bachillerato pero son altamente recomendables para el alumnado que necesite ampliar sus conocimientos matemáticos en cursos posteriores.


\subsubsection{Cómo estudiar matemáticas}

 
Las Matemáticas son una asignatura que no deja indiferente a ningún estudiante. Algunos la aman y otros la odian; siendo este segundo grupo mucho más numeroso que el primero en la mayoría de las ocasiones. Sin embargo, muchos de los estudiantes que odian las matemáticas lo hacen porque no saben cómo estudiarlas para obtener buenos resultados.

Las Matemáticas son una de esas asignaturas en las que las horas de estudio no tienen una relación directa con la nota. Por mucho que hayas estudiado, si no eres capaz de solucionar el problema del examen, estás perdido. No obstante, existen algunas técnicas para aprender matemáticas que pueden hacer que, independientemente de tu nivel, le saques más partido a tu tiempo de estudio y aumentes tus probabilidades de éxito. ¡Hasta es posible que te acabes uniendo al grupo de amantes de las matemáticas!

\emph{Cómo Estudiar Matemáticas:}
 
\begin{enumerate}
	 

\item Práctica, Práctica y Más Práctica

Es imposible aprender matemáticas leyendo y escuchando. Para aprender matemáticas hay que ponerse el mono de trabajo y lanzarse a hacer ejercicios matemáticos. Cuanto más practiques, mejor. Cada ejercicio tiene sus particularidades y es importante haber realizado el máximo número de ejercicios posibles antes de enfrentarnos al examen. Este punto es el más importante de todos y la base del resto de técnicas para estudiar matemáticas de esta lista.

Una vez que entendiste los ejemplos explicados en clase, desarrolla ejercicios o problemas con solución para complementar tus conocimientos. Trata de hacerlos sin ver la respuesta y luego, si ni lo puedes terminar, ayúdate de la solución y sigue desarrollando, siempre preguntándote ?`por qué se realiza dicho paso? Se trata de que poco a poco no tengas necesidad de utilizar las soluciones y llegará el momento, después de unos pocos ejercicios, en que ya no necesites de ellas para desarrollar y entender completamente los problemas

No pienses que escuchando la explicación de muchos ejercicios y problemas, donde un profesor te explica y entiendes un 100$\%$, es lo más importante para aprender Matemáticas. La clave del éxito es \emph{¡PRACTICAR!}

\item Revisa los Errores

Cuando estés practicando con ejercicios, es muy importante que compruebes los resultados y, más importante aún, que te detengas en la parte que has fallado y examines el proceso en detalle hasta asimilarlo. De nada sirve comparar resultados si no sabes en qué te has equivocado. Por eso es conveniente que tengas unos buenos apuntes con problemas resueltos. De esta manera, evitarás cometer los mismos fallos en el futuro. También es recomendable apuntar todos tus fallos y repasarlos repetidamente antes del examen.

Toma buenos apuntes: Sé ordenado, utiliza un solo cuaderno para el curso y escribe claramente usando, en lo posible, tus propias palabras para que puedas entender cuando estudies. Copia todo lo que el profesor diga y escriba en la pizarra, anotando los porqués de cada paso, ya que uno siempre puede olvidar lo escuchado y cuando vuelvas a leer tus apuntes podrás recordarlos rápidamente.

Observa y apunta si el profesor hace hincapié en ciertos puntos basándose en repeticiones, ejemplos, diagramas, comentarios extensos, etc., éstos son casi siempre parte importante de los temas.

\item Domina los Conceptos Clave

¡No intentes aprenderte los problemas de memoria! Los problemas matemáticos pueden tener miles de variantes y particularidades, por lo que es inútil aprendernos problemas de memoria sin entenderlos. Es cambio, es mucho más efectivo dominar los conceptos importantes y el proceso de resolución de los problemas.

Recuerda que las Matemáticas son una asignatura secuencial, por lo que es importante asentar una base firme dominando los conceptos clave y teniendo claras las fórmulas matemáticas esenciales.

\item Consulta tus Dudas

Puede que en muchas ocasiones te sientas atascado en una parte de un problema o que simplemente no entiendas el proceso. Lo común en estos casos es simplemente pasar de ese problema y pasar al siguiente. Sin embargo, es recomendable despejar todas las dudas que tengas en la resolución de un problema.

Por tanto, puede ser buena idea estudiar junto a algún/a compañero/a con el que consultar dudas y trabajar juntos en problemas más complejos. O, mejor todavía, ¿por qué no te unes a un grupo de estudio en el que puedes plantear tus dudas y trabajar colaborativamente? Asimismo, recuerda plantearle al profesor/a las dudas que tengas, ya sea en clase o en una tutoría.

\item Crea un Ambiente de Estudio sin Distracciones.

Las Matemáticas son una asignatura que requiere más concentración que ninguna otra. Un ambiente de estudio adecuado y libre de distracciones puede ser el factor determinante para conseguir resolver ecuaciones o problemas de geometría, álgebra, trigonometría o complejos. Si te gusta estudiar con música, puede ser una buena idea escucharla de fondo para relajarte y favorecer un ambiente de máxima concentración; la música instrumental es la más recomendable en estas ocasiones.

¡Ah, y no olvides que es importante también tener confianza en uno mismo y afrontar el examen sabiendo que te has preparado adecuadamente!

\end{enumerate}

\emph{Empieza a Estudiar Matemáticas Ahora. ¡Es gratis!}

\subsubsection{Guía de lectura}

Los temas marcados con el símbolo  $\divideontimes$ no forman parte de un temario normal de segundo de bachillerato, pero son lo suficientemente importantes para que el/la lector/a les dedique su atención si en su futuro va a  necesitar ampliar su curriculum matemático, si no durante el curso sí durante el verano. Se exponen con la suficiente sencillez como para ser entendidos por cualquier alumno/a de bachillerato. En estos temas no se proponen ejercicios y todos están resueltos. 

\vspace{5mm}
\Large {\textbf{Parte I Álgebra Lineal}}\normalsize{.}

\subsubsection{Estructuras algebráicas  $\divideontimes$}

Empezamos el libro con el capítulo 1 de `estructuras algebraicas' que son la base del álgebra y del concepto de `espacio vectorial' que se verá en un próximo capítulo. No forman parte de un temario ordinario de segundo de bachillerato pero es de interés su conocimiento para seguir con estudios que requieran más bagage matemático. Por la sencillez en la exposición y debido a su carácter meramente introductorio, en este capítulo no habrán ejercicios propuestos, solo presentaremos algunos ejercicios resueltos.

\subsubsection{Sistemas de ecuaciones lineales (SEL). Método de Gauss}

Este tema es altamente importante, asegúrate de entenderlo a la perfección y realizar todos los ejercicios. 

El método aquí desarrollado para la resolución de los sistemas de ecuaciones lineales, `método de Gauss', es una potente y robusta herramienta para enfrentarte a este tipo de problemas, que suele aparecer con frecuencia en matemáticas y en ciencias, no solo en álgebra. En futuros temas analizaremos otros métodos con sus ventajas e inconvenientes frente a éste que ya detallaremos en su momento.

\subsubsection{Matrices}

Se introduce el importantísimo concepto de matriz en matemáticas y se aprende a operar con ellas. Dada la relevancia del tema, es aconsejable dedicarle el tiempo necesario para estar convencido de entenderlo y dominarlo.

\subsubsection{Determinantes}

Son una aplicación del conjunto de matrices cuadradas sobre el conjunto de los números reales: a cada matriz cuadrada se le asigna un número real. 

Asegúrate bien de saber calcular determinantes y de entender sus propiedades, así como su aplicación al calculo de matrices inversas y a la resolución de ecuaciones matriciales.

\subsubsection{SEL: teorema de Rouché}

Comenzamos con la definición de `rango' de una matriz y estudiamos dos métodos de cálculo de rangos: Gauss y orlados. Debes saber calcular rangos antes de aplicar el teorema de Rocuhé, que se explica a continuación.

Se ha intercalado el método de Cramer para la resolución de SEL, que junto con la forma matricial y el método de Gauss son los tres que veremos en este curso.

El tema acaba con la aplicación del teorema de Rouché a la discusión de sistemas dependientes de parámetros y a la eliminación de éstos en las soluciones de los sistemas compatibles indeterminados.

Asegúrate, como en los temas anteriores, de entender bien todos los conceptos que se explican y los ejemplos y ejercicios resueltos que aparecen. Esfuérzate en la resolución de los problemas propuestos, de los que dispones de solución, así como de las cuestiones finales (tienen ayuda)-

\subsubsection{Espacios vectoriales  $\divideontimes$}

En este tema, que está fuera de los temarios ordinarios de bachillerato, se introduce el concepto de `espacio vectorial', se estudian los subespacios y los conceptos de dependencia lineal, sistema generador, base y dimensión. Acabamos el tema con una introducción a los cambios de base en espacios vectoriales.

\subsubsection{Aplicaciones lineales  $\divideontimes$}

Este es otro de los temas que no pertenecen al temario general de bachillerato en que se introduce el importantísimo concepto de homomorfismo o aplicación lineal entre espacios vectoriales. Se habla de los subespacios `núcleo' e `imagen' y de la forma matricial de una aplicación lineal.

\subsubsection{Diagonalización de matrices  $\divideontimes$}

Último tema externo al temario general de bachillerato donde se introduce la diagonalización de matrices cuadradas y su aplicación al cálculo de potencias enésimas de matrices, para ello se introducen y se aprende a calcular los llamados 'valores y vectores propios' de una matriz (aplicación lineal).

\vspace{5mm}
\Large {\textbf{Parte II  Geometría}}\normalsize{.}

\subsubsection{Vectores en el plano}

Tras exponer el concepto de vector fijo del espacio nos centramos en el de vector libre que nos servirá, junto con el conjunto de puntos des espacio, para tratar la geometría analítica que se expondrá en los siguientes temas: los vectores serán los encargados de desplazar a los puntos. 

Se estudia el espacio vectorial de los vectores libres del espacio y el de base ortonormal para definir tres tipos de productos entre vectores: el producto escalar, que nos definirá la métrica del espacio (útil para medir distancias y ángulos); el producto vectorial que nos proporciona la idea de perpendicularidad y, como tercer producto entre vectores veremos una combinación de los dos anteriores, el producto mixto, que nos dará información acerca de volúmenes.

\subsubsection{Rectas y planos}

Definido lo que vamos a entender por rectas y planos y aprendidas las distintas formas de expresarlos, pasamos a estudiar sus características afines, sus posiciones relativas.

\subsubsection{Problemas métricos}

Dotando al espacio afín (rectas y planos) de una métrica, producto escalar, aprenderemos en este tema a calcular ángulos y distancias entre puntos, rectas y planos.

\subsubsection{Superfícies $\divideontimes$}

Acabamos la geometría con una somera descripción de las superficies en $\mathbb R^3$. Hablamos de la esfera y el cilindros y mencionamos las coordenadas esféricas y cilíndricas. Nombramos las superficies cónicas y las cuádricas.

\vspace{5mm}
\Large {\textbf{Parte III  Apéndices}}\normalsize{.}

\subsubsection{Apéndices}

El Apéndice \ref{simbolosmat} está dedicado a los símbolos matemáticos básicos.

En el apéndice \ref{sumatorioproductorio} se puede encontrar la definición y propiedades de los `sumatorios y productorios', operadores muy usados en matemáticas.
$\displaystyle \sum_{i=1}^n a_i\; \quad \prod_{i=1}^n a_i$

El apéndice \ref{inducción} se enuncia el `principio de inducción' y se exponen varios ejemplos de su aplicación en demostraciones matemáticas.

El apéndice \ref{Vandermonde} se estudia el determinante de Vandermonde de orden-$n$ y se ve una de sus muchas aplicaciones, el `polinomio inerpolador'.


El apéndice \ref{VectoresDistintosSistemasCoordenadas} muestra la forma de expresar un `vector de posición' de un punto en coordenadas cilíndricas y esféricas.

\subsubsection{Agradecimientos y licencia}

\centering{
\fcolorbox{black}{fondoblau}{
\parbox{0.95\textwidth}{
	\textit{Este material es un conjunto de apuntes personales, que comparto gratuitamente en la red, basados en mi experiencia como profesor, varios textos citados anteriormente y webs de internet. Si hay algún contenido que no he incluido correctamente, hacédmelo saber por e-mail y lo editaré como se pida.  También se agradecería la comunicación de la detección de cualquier error.}
}}}
\justify



\vspace{5mm}
\emph{Este documento se comparte bajo licencia `Attribution-NonCommercial 4.0 International (CC BY-NC 4.0)'}
\vspace{5mm}

\begin{multicols}{2}
\begin{figure}[H]
	\centering
	\includegraphics[width=.4
	\textwidth]{imagenes/imagenes00/licencia.png}
\end{figure}
\begin{figure}[H]
	\centering
	\includegraphics[width=.3
	\textwidth]{imagenes/firma.png}
\end{figure}
\end{multicols}




%\begin{figure}[H]
	%\centering
	%\includegraphics[width=1\textwidth]{imagenes/imagenes00/xiste00.png}
%\end{figure}



